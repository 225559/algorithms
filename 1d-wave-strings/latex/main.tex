\documentclass{article}
\usepackage[utf8]{inputenc}
\usepackage{amsmath}
\usepackage[english]{babel}

\usepackage{color}

\usepackage{listings}[]
\definecolor{codegreen}{rgb}{0,0.6,0}
\definecolor{codegray}{rgb}{0.5,0.5,0.5}
\definecolor{codepurple}{rgb}{0.58,0,0.82}
\definecolor{backcolour}{rgb}{0.95,0.95,0.92}

\lstdefinestyle{mystyle}{
    backgroundcolor=\color{backcolour},   
    commentstyle=\color{codegreen},
    keywordstyle=\color{magenta},
    numberstyle=\tiny\color{codegray},
    stringstyle=\color{codepurple},
    basicstyle=\small,
    breakatwhitespace=false,         
    breaklines=true,                 
    captionpos=b,                    
    keepspaces=true,                 
    numbers=left,                    
    numbersep=5pt,                  
    showspaces=false,                
    showstringspaces=false,
    showtabs=false,                  
    tabsize=2
}

\lstset{style=mystyle}

\usepackage{biblatex}
\addbibresource{ref.bib}

\title{Simulating 1D Waves in Python}
\author{Sorn Zupanic Maksumic}
\date{{\small February 2021}}

\begin{document}


\maketitle

\newpage
\tableofcontents

\newpage
\section{Overview}
This document shows how to solve the 1D wave equation (Equation \ref{eq:1d-wave}) in Python. The solution is approximated with the central finite difference method (also known as the Crank-Nicholson scheme). The algorithm description is influenced by \cite{hpl-decay} and \cite{hpl-wave}.

\begin{equation}
    \frac{\partial^2}{\partial t^2}u(x,t) = c^2\frac{\partial^2}{\partial x^2}u(x,t)
    \label{eq:1d-wave}
\end{equation}

\subsection{Terminology}
It's called a 1D wave equation even though $u(x,t)$ has two inputs: $x$ and $t$. This is because we only count the number of inputs in the spacial domain.

\noindent\\The output $u(x,t)$ is the $y$-coordinate in space, for the given $x$ and $t$.

\subsection{Dependencies}
The Python script depends on the following 3\textsuperscript{rd} party libraries:
\begin{itemize}
    \item Matplotlib
    \item Numpy
\end{itemize}

\subsection{Result}
The program outputs an animated GIF file. The animation shows six 1D waves with different velocities.


% \noindent\\Equation \ref{eq:1d-wave} is often written as Equation \ref{eq:1d-wave-short}:

% \begin{equation}
%     u_{tt} = c^2 u_{xx}
%     \label{eq:1d-wave-short}
% \end{equation}

\newpage
\section{Algorithm Description}
This document uses $u_i^{n}$ as a concise notation for $u(x_i, t_n)$. Similarly, $u_{ij}^n$ is a concise notation for $u(x_i, y_j, t_n)$.

\subsection{Initial Conditions}
The Initial Condition (IC) is shown in Equation \ref{eq:ic}. It returns $u$ for all $x_i$ when $t=0$. In other words, it returns the start position for each $x_i$. The equation set given in Equation \ref{eq:ic} produces a "triangle-shaped" IC.

\begin{equation}
    I(x_i)=u_i^0=u(x_i, t_0)=
        \begin{cases}
            u(i\Delta x,0)=\frac{A}{\bar x}x, 0 \leq x \leq 1\\
            u(i\Delta x,0)=\frac{A}{L-\bar x}(L-x), 1 \leq x < L
        \end{cases}
    \label{eq:ic}
\end{equation}

\noindent The IC for the first-order derivative is zero (see Equation \ref{eq:icd}). In other words, there is no initial velocity. This is important in a later step, because it lets us approximate Equation \ref{eq:icd} with a first-order partial central difference and get Equation \ref{eq:icd-sub}.

\begin{equation}
    \frac{\partial}{\partial t}u(x,0)=0
    \label{eq:icd}
\end{equation}

\begin{equation}
    \frac{\partial}{\partial t}u(x_i,t_0) \approx 
    \frac{u_i^1-u_i^{-1}}{2\Delta t} = 0 \implies
    u_i^1 = u_i^{-1}
    \label{eq:icd-sub}
\end{equation}

\subsection{Boundary Conditions}
The boundary conditions are shown in Equation \ref{eq:bc}, where $L$ is the length of the wave. In other words, the wave is \textit{fixed} at both ends.

\begin{equation}
    u(0,t)=0\ \wedge\ u(L,t)=0
    \label{eq:bc}
\end{equation}

\subsection{Central Finite Difference}
% Equation \ref{eq:1d-wave} is approximated with second-order central finite differences, and rewritten as Equation \ref{eq:cfdm}.

% \begin{equation}
%     u_i^{n+1}=C^2(u_{i+1}^n-2u_i^n+u_{i-1}^n)+2u_i^n-u_i^{n-1}
%     \label{eq:cfdm}
% \end{equation}

Start with the 1D wave equation (Equation \ref{eq:1d-wave}) and replace both sides of the equation with second-order central finite differences. The second-order central finite differences are given as Equation \ref{eq:cfdm-approx-dtt} and \ref{eq:cfdm-approx-dxx}.

\begin{equation}
    \frac{\partial^2}{\partial t^2}u(x_i,t_n)\approx \frac{u_i^{n+1}-2u_i^n+u_i^{n-1}}{\Delta t^2}
    \label{eq:cfdm-approx-dtt}
\end{equation}

\begin{equation}
    \frac{\partial^2}{\partial x^2}u(x_i,t_n)\approx \frac{u_{i+1}^{n}-2u_i^n+u_{i-1}^{n}}{\Delta x^2}
    \label{eq:cfdm-approx-dxx}
\end{equation}

\newpage
\noindent\\Result:

\begin{equation}
    \frac{u_i^{n+1}-2u_i^n+u_i^{n-1}}{\Delta t^2}
    = c^2\left(
    \frac{u_{i+1}^{n}-2u_i^n+u_{i-1}^{n}}{\Delta x^2}
    \right)
    \label{}
\end{equation}

\noindent\\Solve for $u_i^{n+1}$:

\begin{equation}
    u_i^{n+1} = C^2(u_{i+1}^n - 2u_i^n + u_{i-1}^n) + 2u_i^n-u_i^{n-1}
    \label{eq:cfdm}
\end{equation}

\noindent\\Here, $C$ is Courant's number:
\begin{equation}
    C=c\frac{\Delta x}{\Delta t}
\end{equation}

\noindent\\The "problem" with Equation \ref{eq:cfdm} is that we do not have $u_i^{n-1}$. Therefore, we use Equation \ref{eq:icd-sub} to substitute $u_i^{n-1}$ with $u_i^{n+1}$, do some simple algebra, and get Equation \ref{eq:firstfunc}:

\begin{equation}
    u_i^{n+1}=\frac{1}{2}C^2\left(u_{i+1}^n-2u_i^n+u_{i-1}^n\right)+u_i^n
    \label{eq:firstfunc}
\end{equation}

\noindent\\We use the IC to get all $u_i^0$. Then, solve $u_i^{1}$ with Equation \ref{eq:firstfunc}. Now, that we have $u_i^{n-1}$, use Equation \ref{eq:cfdm} to solve the rest.

\noindent\\That's it!

\newpage
\section{Code Explanation}
Here, some parts of the code are explained. The code is commented with further explanations.

\subsection{Solver}
The first loop applies Equation \ref{eq:ic} to find $u_i^0$ for all $i$:

\begin{lstlisting}[language=Python, caption=Initial Condition]
for i in range(0, Nx+1):
    u[0][i] = I(i*dx, A, s, L)
\end{lstlisting}

\noindent\\The second loop applies Equation \ref{eq:icd-sub} to find $u_i^1$ for all $i$:

\begin{lstlisting}[language=Python, caption=Initial Condition]
for i in range(1, Nx):
    u[1][i] = 0.5*(C**2)*(u[0][i+1]-2*u[0][i]+u[0][i-1]) + \
              u[0][i]
\end{lstlisting}

\noindent\\The third loop applies Equation \ref{eq:cfdm} to solve the rest of the 1D wave equation.

\begin{lstlisting}[language=Python, caption=Initial Condition]
for n in range(1, Nt):
    for i in range(1, Nx):
        u[n+1][i] = (C**2)*(u[n][i+1]-2*u[n][i]+u[n][i-1]) + \
                    2*u[n][i]-u[n-1][i]
\end{lstlisting}

\subsection{Animation}
The reason for first looping over all the time steps and \textit{then} over all the 1D waves, is to group the correct animations together. If you do it the other way around, it only display one wave at a time: first displaying the entire animation for the first 1D wave, then the second, ...

\begin{lstlisting}[language=Python, caption=Initial Condition]
for n in range(T):
    plts_tmp = []
    for i in range(len(li)):
        u, x, color = svl[i][0], svl[i][1], svl[i][2]
        p = plt.plot(x, u[n][:] + offset * i, color)
        plts_tmp.append(*p)
    plts.append(plts_tmp)
\end{lstlisting}

\noindent\\But by grouping the correct animations together it displays the animations for all the 1D waves at the same time.

\newpage
\printbibliography

\end{document}
